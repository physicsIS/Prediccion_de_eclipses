\setcounter{section}{0}
\begin{singlespace}%Inicio de espaciado simple.

\huge{\textbf{Objetivos}} 

\normalsize

\section{Objetivo General}
\hspace{1 cm} Construir un modelo teórico para la reconstrucción de accidentes de tránsito donde se consideren geometrías complejas vehiculares y colisiones múltiples, partiendo del trabajo de la empresa WARDEN: Investigaciones Aplicadas.

\section{Objetivos Específicos}
\normalsize

\begin{itemize}
    \item Presentar un conjunto de las geometrías apropiado para aproximar diferentes tipos de vehículos.
    
    \item Determinar el sistema de ecuaciones diferenciales que describe el comportamiento de los entes de interés en colisiones únicas o múltiples en geometrías complejas.
    
    \item Comparar los resultados esperados de la aproximación teórica planteada con un caso de colisión de interés, mostrando la validez o no de la aproximación.

     \item Elaborar un compendio con la teoría mecánica necesaria para desarrollar un modelo de accidentes vehículares basado en geometrías complejas.
\end{itemize}

\end{singlespace}