\setcounter{section}{0}
\begin{singlespace}

\huge{\textbf{Metodología de la Investigación}}
\normalsize

\section{Metodología}
\hspace{1 cm}  Para la investigación se cuenta con la guía de un tutor perteneciente tanto a la Universidad de Costa Rica como a la empresa WARDEN: Investigaciones Aplicadas, Bryan Hidaldo.


\hspace{1 cm} Como tal, esta es primera parte de la investigación posee un caracter más que todo teórico y tomando en cuenta que la teoría de la dinámica vehicular es un campo de la mecánica ya bien establecido, no se partirá de cero en ningún sentido a lo largo del desarrollo de este texto. Por lo que el trabajo, en su mayoría se invertirá en el estudio de las bases de la mecánica necesarias para avanzar al estudio de la dinámica vehicular de forma general para, posteriormente, focalizarse en comprender la teoría utilizada en el modelo de reconstrucción de accidentes de WARDEN. Una vez logrado esto, se deberá realizar una revisión de la teoría necesaria para trabajar con geometrías vehiculares complejas y construir tales estructuras geométricas para realizar la extracción de los datos necesarios para el modelado de este tipo de estructuras vehiculares.

\hspace{1 cm} Para este momento, aún no se habrían realizado modificaciones en la estructura del modelo de WARDEN, no obstante, la primera modificación al modelo en cuestión se llevaría a cabo con el agregado de las consideraciones de frenado de uno o varios de los vehículos implicados en colisiones. El principal propósito de realizar esta modificación, más allá de incrementar los casos para los que el modelo es válido, es que permitirá verificar la compresión de la teoría de dinámica vehicular y del modelo teórico de utilizado por WARDEN.

\hspace{1 cm} Una vez  se haya completado todo lo anterior, se podría dar paso a lo grueso de la investigación, el desarrollo del modelo que abarque colisiones múltiples. Para esto será necesario establecer condiciones de contacto para la colisión de los vehículos considerándolos como sólidos rígidos y estudiar el comportamiento de cuerpos durante colisiones múltiples y, de esta manera terminar el desarrollo del modelo de interés con una validación inicial del modelo generado. Esta validación se realizaría haciendo uso de colisiones controladas y verificando si el comportamiento esperado según el modelo creado discrepa o no del comportamiento según las colisiones reales. Por último es preciso agregar que, para facilitar estudios similares a este, se desarrollará un compendio propio a esta investigación como un resumen o una guía de la teoría utilizada a lo largo de este desarrollo.



\hspace{1 cm} A contiuación se presenta el trabajo previo que se plantea antes de comenzar formalmente con esta investigación:
\section{Trabajo previo}
\begin{enumerate}
\item Estudio intensivo de mecánica de forma general a partir de bibliografía apropiada.
\item Investigación y estudio sobre mecánica de cuerpos rígidos simples y con formas complejas haciendo uso de la bibliografía obtenida.
\item Estudiar en profundidad la teoría empleada en dinámica de vehículos aplicada a reconstrucción de accidentes vehículares, utilizando los libros recomendados por el tutor de la investigación.
\item Analizar la teoría empleada por WARDEN en sus modelos teóricos actuales y examinar los modelos teóricos en sí.
\item Generar una pequeña lista de los tipos de vehículos más comunes en Costa Rica.
\end{enumerate}

\hspace{1 cm} Seguidamente, se tienen a grandes rasgos los pasos que se van a seguir durante la investigación para la construcción del modelo de interés:

\section{Procedimiento}
\begin{enumerate}
\item Comenzar el compendio con una teoría base e ir ingresando la teoría utilizada en cada paso.
\item  A partir de la lista de vehículos más comunes en el país, generar modelos geométricos de los mismos, utilizando figuras simples como pueden ser paralelepípedos, cuñas,...
\item Utilizar las contrucciones geométricas anteriores dotandolas de diferentes densidades para un posterior cálculo de los centros de masa y momentos de inercia de cada vehículo.
\item Estudiar el caso de que uno o ambos vehículos frenen al momento de la colisión.
\item Integrar este caso al modelo teórico de WARDEN.
\item Añadir las geometrías complejas al modelo en construcción.
\item Comparar el modelo de WARDEN original con el modelo que incluye geométricas complejas y frenado.
\item Estudiar el caso de colisiones vehiculares múltiples.
\item Construir un modelo teórico que describa este tipo de situaciones al colisionar.
\item Estudiar el sistema de ecuaciones obtenido para describir la situación de colisiones múltiples y verificar si el comportamiento del mismo es capaz de describir lo que ocurre en una colisión múltiple controlada. De no lograr la descripción de este caso, regresar al paso 8.
\item Finalizar el compendio.
\end{enumerate}

\end{singlespace}