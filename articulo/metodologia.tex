\setcounter{section}{0}
\begin{singlespace}

\huge{\textbf{Metodología de la Investigación}}
\normalsize

\section{Metodología}
\hspace{1 cm}  Para la realización de esta investigación se cuenta con un gran número de integrantes en diversos ciclos del Bachillerato en Física, por lo que se trabajará tomando en cuenta los diferentes niveles de conocimiento de los mismos y promoviendo el aprendizaje mientras se lleva a cabo la investigación.


\hspace{1 cm} Como tal,

\hspace{1 cm} 

\hspace{1 cm} 



\hspace{1 cm} A contiuación se presenta el trabajo previo que se plantea antes de comenzar formalmente con esta investigación:
\section{Trabajo previo}
\begin{enumerate}
    \item Busqueda de antecedentes en la predicción de eclipses.
    \item Obtener una base de datos de los próximos eclipses, donde se incluya fecha, hora, lugar, duración y tipo de eclipse.
    \item Investigar sobre paquetes o bibliotecas para la visualización de trayectorias sobre la superficie terrestre.
\end{enumerate}

\hspace{1 cm} Seguidamente, se tienen a grandes rasgos los pasos que se van a seguir durante la investigación para la construcción del modelo de interés:

\section{Procedimiento}
\begin{enumerate}
    \item
\end{enumerate}

\end{singlespace}