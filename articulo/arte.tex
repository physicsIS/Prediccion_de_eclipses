\setcounter{section}{0}
\begin{singlespace}

\huge{\textbf{Estado del Arte}}

\normalsize
\section{Artículos de interés}

\hspace{1 cm}A continuación se mencionan en brevedad artículos y otros textos que se consideran de interés para el desarrollo de esta investigación.

\hspace{1 cm} Comenzando con \citet{Yang} quienes realizan una revisión a modo de resumen de la teoría desarrollada en la dinámica vehicular en los últimos años, dando una guía de los conceptos y relaciones físicas que se deben de comprender para la realización de modelos teóricos más modernos en cuento a la simulación de vehículos en diferentes situaciones. El artículo aclara que el peso o masa suspendida, la suspension (como sistema de resorte-armortiguador) y masa no suspendida (ruedas) como parte escencial en la construcción de un modelado vehicular,  además de añadir ideas sobre la dinámica de manejo y conducción. Este artículo es de interés para contextualizar y dirigir el estudio que se debe de realizar para el desarrollo de esta investigación.

\hspace{1 cm} Seguidamente, \citet{Steffan2009} presenta un trabajo similar al anterior, no obstante este es mucho más extenso, mostrando con cierto grado de detalle técnicas utilizadas para el modelado vehicular en recontrucción de accidentes. \citet{Steffan2009} describe las bases con las que se establece un modelo de predicción de la trayectoria vehicular sobre la cinemática de tiempo discreto y un vehículo como un cuerpo rígido expuesto a fuerzas externas. Luego de esto, se mencionan los diferentes modelos a los cuales se puede recurrir para el modelado, además de incluir algunas consideraciones físicas necesarias para la construcción  de estos diferentes modelos.  Su compendio toca la dificultades que se pueden tener al escoger ciertos métodos/modelos para la reconstrucción del accidente estudiado, dado que algunos de estos requieren variables que dificilmente son conocidas a la hora de la reconstrucción. El trabajo de \citet{Steffan2009} servirá de guía junto con el anterior para dirigir la formulación del modelo teórico a construir en la presente investigación.

\hspace{1 cm} Por otro lado, se tiene \citet{Risac} artículo en el cual se agrupan 5 diferentes colisiones controladas con el objetivo de probar una metología de recontrucción de accidentes denominada: ``PC-Crash''. En el artículo se detalla por completo las condiciones relevantes para la colisión antes, durante y después de la misma, permitiendo una posterior revisión de ``PC-Crash'' como modelo coherente de reconstrucción. Esto último será de gran importancia en esta investigación, dado que al tener los datos reales de las colisiones y los de otro modelo de reconstrucción de colisiones se podrá validar si el modelo generado describe o no las colisiones para el caso de colisiones múltiples de forma cualitativa y posteriormente, de forma cuantitativa en la segunda parte de la investigación, confirmando si el modelo teórico es efectivo en la descripción de las colisiones de interés.

\hspace{1 cm} Ahora bien, en \citet{Hogue} se habla sobre la representación de geometrías arbitrarias y detección de contacto entre ellas. El artículo toca diferentes tipos de cuerpos de estudio en 2 y 3 dimensiones utilizando diferentes formas de respresentarlos, como pueden ser los poliedros, las representaciones por medio de funciones continuas o la representación utilizando funciones discretas para los objetos en 3 dimensiones y como se plantea la detención de colisiones en cada caso con sus respectivas ventajas y desventajas. Como tal, este artículo es útil para identificar el tipo de métodos que pueden ser utiles para reconstruir la geométria vehicular y los prodecimientos para identificar la colisión según el tipo de estructura que se utilice en la reconstrucción.

\hspace{1 cm} Siguiendo con \citet{Orasanu2010} presenta una alternativa para el cálculo de momentos de inercia de cuerpos en 2 o 3 dimensiones de forma más sencilla a partir de reducciones en la dimensión del cuerpo estudiado, terminando siempre en un arreglo de partículas que poseen las mismas propiedades inerciales (Mismo centro de masa y mismo tensor de inercia).  Lo que presenta este artículo es sumamente útil para el desarrollo de la investigación al plantear un método bastante sencillo para el cálculo del centro de masa y momentos de inercia de sólidos. Estos objetos se podrían colocar en grandes conjuntos y conformar una aproximación de la estructura del vehículo que permita realizar los cálculos mencionados con anterioridad y aproximar con mayor exactitud el centro de masa y el tensor de inercia del cuerpo de estudio real, los diferentes vehículos de tránsito.

\hspace{1 cm} Por último, se presenta a \citet{Wardenia}, el cual es un compendio con el desarrollo teórico que actualmente utiliza WARDEN: Investigaciones Aplicadas en el modelo de reconstrucción de colisiones. El compendio menciona muy brevemente la teoría relacionada a la dinámica vehicular y la dinámica de neumáticos, teoría para casos de colisión con peatón, entre otros. Al igual que otros artículos mencionados al inicio de esta sección, este texto se utilizará como guía para la formulación del modelo teórico que se desarrolla en esta investigación, siendo precisamente este texto la base teórica más importante, en un inicio, para poder comprender y construir sobre el modelo que utiliza WARDEN en la reconstrucción de accidentes de tránsito.


\end{singlespace}