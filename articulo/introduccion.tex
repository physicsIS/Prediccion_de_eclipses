\setcounter{section}{1}
\begin{singlespace}

    \huge{\textbf{Introducción}} 

    \normalsize

    \subsection{Motivación del proyecto}
    La predicción de eclipses es un tema de interés, tanto científico como cultural, ya que a lo largo de la historia los eclipses han
    fascinado a la humanidad provocando asombro e interés ante la inmensidad del cosmos. Estos fenómenos han sido interpretados de diversas
    maneras en diferentes culturas, desde presagios divinos hasta eventos astrológicos. \\
    Por lo tanto, la capacidad de predecir este tipo de eventos no solo demuestra el avance en nuestras herramientas matemáticas y
    científicas, sino que también refleja nuestra comprensión del universo y las leyes que lo rigen.\\
    En la actualidad, la precisión de los eclipses se ha vuelto parte esencial en nuestras vidas, no solo de interes para la astronomia,
    la investigación academica y la comprensión de fenómenos atmosféricos, sino tambien para la planificación de eventos turísticos y
    culturales, ya que muchas personas viajan a lugares específicos para experimentar estos fenómenos, lo cual promueve la economía local y contribuye
    a la apreciación y divulgación cientifica.\\ 
    Este anteproyecto propone desarrollar un modelo computacional para la predicción y visualización de eclipses, utilizando herramientas
    matemáticas y simulaciones en 3D en Python.\\ A través de este enfoque, no solo buscamos replicar los métodos históricos de predicción,
    sino desarrollar un metodo propio de predicción y contrastar con datos de la NASA. 
    
\end{singlespace}